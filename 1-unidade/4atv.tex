  
\documentclass[a4paper,12pt]{article}
\author{}
\date{}
\usepackage[papersize={216mm,330mm},tmargin=20mm,bmargin=20mm,lmargin=20mm,rmargin=20mm]{geometry}
\usepackage[brazil]{babel}
\usepackage[utf8]{inputenc}
\usepackage{amsmath,amssymb} %,mathabx}%\for eqref
\usepackage{lscape}
\usepackage{graphicx}
\usepackage[colorinlistoftodos]{todonotes}
\usepackage{fancyhdr}
\usepackage{tasks}
\usepackage{float}
\usepackage{multicol}
\usepackage{color}
\usepackage{ragged2e} % justifying
\usepackage{colortbl}
\usepackage{cancel}
\usepackage{stackengine}
\usepackage{mathtools}
\usepackage{tikz}   %create images
\usepackage{tkz-fct}    %create graphics

%====================================================
%============= INFORMACAO DE ENTRADA=================
%====================================================

\def\nomedoaluno{José Manoel Freitas da Silva}
\def\matricula{40028922}

%IMD1002 - Análise Combinatória
%IMD0024 - Cálculo 1
%IMD0034 - Vetores e Geometria Analítica
\def\coddisciplina{IMD0034}
\def\nomedisciplina{Vetores e Geometria Analítica}
\def\codturma{VGA20251T2}
\def\codatividade{U1\_VGATRAB40}

%====================================================
%====================== FIM =========================
%====================================================

\pagestyle{fancy}
\fancyhf{}
\lhead{Aluno: \nomedoaluno\\ Matrícula: \matricula}
\chead{\thepage}
\rhead{Instituto Metrópole Digital - UFRN \\
Turma \codturma}
\lfoot{\nomedisciplina}
\cfoot{Atividade \codatividade}
\rfoot{Prof. Samyr Jácome}

\title{
\vspace{-2cm}
\Large \textbf{Universidade Federal do Rio Grande do Norte}\\ 
Instituto Metrópole Digital \\ 
\coddisciplina $\;$ - \nomedisciplina \\ 
TURMA \codturma \\
\vspace{5mm} \Large\textbf{ATIVIDADE \codatividade} \\
\normalsize Natal-RN, \today\\
\vspace{0.7cm} \large \textit{Prof. Samyr Jácome}\\

\justifying
\vspace{0.5cm} \hspace{-0.82cm}
\begin{minipage}{.6\linewidth}
    \large \textbf{Aluno:} \nomedoaluno
\end{minipage}
\begin{minipage}{.4\linewidth}
    \begin{flushright}
        \large \textbf{Matrícula:} \matricula
    \end{flushright}
\end{minipage}
}

% Tirar a identação do paragrafo no texto
\def\tirarident{\setlength{\parindent}{0cm}} % padrão 15pt.

\setlength{\marginparwidth}{2cm}

%---------------------------------------------------------------
%---------------COMANDO PARA SETA DE ESCALONAMENTO--------------
%---------------------------------------------------------------

\newcommand{\seta}[3][-0.05cm]{%
  \stackon[#1]{
    $\xrightarrow{\mathmakebox[\setawidth]{}}$}{
    \scriptsize{$#2 \,\leftarrow\, #3$}
    }
}

\newcommand{\troca}[3][-0.05cm]{%
  \stackon[#1]{
    $\xrightarrow{\mathmakebox[\setawidth]{}}$}{
    \scriptsize{$#2 \,\leftrightarrow\, #3$}
    }
}

\newlength{\setawidth}% row operation width
\AtBeginDocument{\setlength{\setawidth}{2.0cm}}

%---------------------------------------------------------------
%--------------------------FIM----------------------------------
%---------------------------------------------------------------

\begin{document}
\maketitle

\vspace{-2cm}
\section*{Exercícios}

\tirarident

\section*{Questão 3 - Página 117}

Matriz original:
\[
A = \begin{bmatrix}
-1 & 5 & 2 \\
0 & 2 & -1 \\
-3 & 1 & 1
\end{bmatrix}
\]

Aplicamos a operação:
\[
\seta{C_2}{C_2 + 2C_3}
\]

Matriz resultante:
\[
\begin{bmatrix}
-1 & 9 & 2 \\
0 & 0 & -1 \\
-3 & 3 & 1
\end{bmatrix}
\]

Escolheremos a segunda linha para calcular o determinante por Laplace.

Como $a_{2,1}$ e $a_{2,2}$ são zeros, os cofatores correspondentes são nulos. Calculamos apenas:

\[
\text{cofator } a_{2,3} = -1 \cdot (-1)^{2+3} \cdot 
\begin{vmatrix}
-1 & 9 \\
-3 & 3
\end{vmatrix}
= -1 \cdot (-1) \cdot (-1 \cdot 3 - (-3) \cdot 9) = 24
\]

\[
\Rightarrow \det(A) = 24
\]


\section*{Questão 5 - Página 117}

Matriz original:
\[
A = \begin{bmatrix}
3 & 0 & -1 \\
1 & 1 & 1 \\
0 & 4 & 2
\end{bmatrix}
\]

Aplicamos a operação:
\[
\seta{C_1}{C_1 + 3C_3}
\]

Matriz resultante:
\[
\begin{bmatrix}
0 & 0 & -1 \\
4 & 1 & 1 \\
6 & 4 & 2
\end{bmatrix}
\]

Usamos a primeira linha para expansão de cofatores.

Como $a_{1,1}$ e $a_{1,2}$ são zeros, calculamos apenas:

\[
\text{cofator } a_{1,3} = -1 \cdot (-1)^{1+3} \cdot 
\begin{vmatrix}
4 & 1 \\
0 & 4
\end{vmatrix}
= -1 \cdot (4 \cdot 4 - 0) = -16
\]

\[
\Rightarrow \det(A) = -16
\]

\section*{Questão 19 - Página 117}

Matriz original:
\[
A = \begin{bmatrix}
-1 & 5 & 2 \\
0 & 2 & -1 \\
-3 & 1 & 1
\end{bmatrix}
\]

Do exercício 3, sabemos que $\det(A) = 24$.

Pelo Teorema 2.3.6, a inversa é:
\[
A^{-1} = \frac{1}{\det(A)} \cdot \text{adj}(A) = \frac{1}{24} \cdot \text{adj}(A)
\]

\subsection*{Cálculo dos Cofatores}

\begin{align*}
C_{1,1} &= (+) \cdot \begin{vmatrix} 2 & -1 \\ 1 & 1 \end{vmatrix} = 3 &
C_{1,2} &= (-) \cdot \begin{vmatrix} 0 & -1 \\ -3 & 1 \end{vmatrix} = 3 &
C_{1,3} &= (+) \cdot \begin{vmatrix} 0 & 2 \\ -3 & 1 \end{vmatrix} = 6 \\
C_{2,1} &= (-) \cdot \begin{vmatrix} 5 & 2 \\ 1 & 1 \end{vmatrix} = -3 &
C_{2,2} &= (+) \cdot \begin{vmatrix} -1 & 2 \\ -3 & 1 \end{vmatrix} = 5 &
C_{2,3} &= (-) \cdot \begin{vmatrix} -1 & 5 \\ -3 & 1 \end{vmatrix} = -14 \\
C_{3,1} &= (+) \cdot \begin{vmatrix} 5 & 2 \\ 2 & -1 \end{vmatrix} = -9 &
C_{3,2} &= (-) \cdot \begin{vmatrix} -1 & 2 \\ 0 & -1 \end{vmatrix} = -1 &
C_{3,3} &= (+) \cdot \begin{vmatrix} -1 & 5 \\ 0 & 2 \end{vmatrix} = -2 \\
\end{align*}

\subsection*{Matriz Adjunta (Transposta da Matriz de Cofatores)}
\[
\text{adj}(A) = 
\begin{bmatrix}
3 & 3 & 6 \\
-3 & 5 & -14 \\
-9 & -1 & -2
\end{bmatrix}
\]

\subsection*{Matriz Inversa}
\[
A^{-1} = \frac{1}{24}
\begin{bmatrix}
3 & 3 & 6 \\
-3 & 5 & -14 \\
-9 & -1 & -2
\end{bmatrix}
=
\begin{bmatrix}
\frac{1}{8} & \frac{1}{8} & \frac{1}{4} \\
-\frac{1}{8} & \frac{5}{24} & -\frac{7}{12} \\
-\frac{3}{8} & -\frac{1}{24} & -\frac{1}{12}
\end{bmatrix}
\]


\end{document}