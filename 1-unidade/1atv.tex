
\documentclass[a4paper,12pt]{article}
\author{}
\date{}
\usepackage[papersize={216mm,330mm},tmargin=20mm,bmargin=20mm,lmargin=20mm,rmargin=20mm]{geometry}
\usepackage[brazil]{babel}
\usepackage[utf8]{inputenc}
\usepackage{amsmath,amssymb} %,mathabx}%\for eqref
\usepackage{lscape}
\usepackage{graphicx}
\usepackage[colorinlistoftodos]{todonotes}
\usepackage{fancyhdr}
\usepackage{tasks}
\usepackage{float}
\usepackage{multicol}
\usepackage{color}
\usepackage{ragged2e} % justifying
\usepackage{colortbl}
\usepackage{cancel}
\usepackage{stackengine}
\usepackage{mathtools}
\usepackage{tikz}   %create images
\usepackage{tkz-fct}    %create graphics

%====================================================
%============= INFORMACAO DE ENTRADA=================
%====================================================

\def\nomedoaluno{José Manoel Freitas da Silva}
\def\matricula{40028922}

%IMD1002 - Análise Combinatória
%IMD0024 - Cálculo 1
%IMD0034 - Vetores e Geometria Analítica
\def\coddisciplina{IMD0034}
\def\nomedisciplina{Vetores e Geometria Analítica}
\def\codturma{VGA20251T2}
\def\codatividade{U1\_VGATRAB10}

%====================================================
%====================== FIM =========================
%====================================================

\pagestyle{fancy}
\fancyhf{}
\lhead{Aluno: \nomedoaluno\\ Matrícula: \matricula}
\chead{\thepage}
\rhead{Instituto Metrópole Digital - UFRN \\
Turma \codturma}
\lfoot{\nomedisciplina}
\cfoot{Atividade \codatividade}
\rfoot{Prof. Samyr Jácome}

\title{
\vspace{-2cm}
\Large \textbf{Universidade Federal do Rio Grande do Norte}\\ 
Instituto Metrópole Digital \\ 
\coddisciplina $\;$ - \nomedisciplina \\ 
TURMA \codturma \\
\vspace{5mm} \Large\textbf{ATIVIDADE \codatividade} \\
\normalsize Natal-RN, \today\\
\vspace{0.7cm} \large \textit{Prof. Samyr Jácome}\\

\justifying
\vspace{0.5cm} \hspace{-0.82cm}
\begin{minipage}{.6\linewidth}
    \large \textbf{Aluno:} \nomedoaluno
\end{minipage}
\begin{minipage}{.4\linewidth}
    \begin{flushright}
        \large \textbf{Matrícula:} \matricula
    \end{flushright}
\end{minipage}
}

% Tirar a identação do paragrafo no texto
\def\tirarident{\setlength{\parindent}{0cm}} % padrão 15pt.

\setlength{\marginparwidth}{2cm}

%---------------------------------------------------------------
%---------------COMANDO PARA SETA DE ESCALONAMENTO--------------
%---------------------------------------------------------------

\newcommand{\seta}[3][-0.05cm]{%
  \stackon[#1]{
    $\xrightarrow{\mathmakebox[\setawidth]{}}$}{
    \scriptsize{$#2 \,\leftarrow\, #3$}
    }
}

\newcommand{\troca}[3][-0.05cm]{%
  \stackon[#1]{
    $\xrightarrow{\mathmakebox[\setawidth]{}}$}{
    \scriptsize{$#2 \,\leftrightarrow\, #3$}
    }
}

\newlength{\setawidth}% row operation width
\AtBeginDocument{\setlength{\setawidth}{2.0cm}}

%---------------------------------------------------------------
%--------------------------FIM----------------------------------
%---------------------------------------------------------------

\begin{document}
\maketitle

\vspace{-2cm}
\section*{Exercícios}

\tirarident

\textbf{Questão 5 item a, pág. 36:}\\


Sejam as matrizes
\[
A = \begin{bmatrix}
3 & 0 \\
-1 & 2 \\
1 & 1 
\end{bmatrix}
\quad \text{e} \quad
B = \begin{bmatrix}
4 & -1 \\
0 & 2
\end{bmatrix}.
\]
Como \(A\) é de ordem \(3 \times 2\) e \(B\) de ordem \(2 \times 2\), o produto
\[
C = A \times B
\]

é uma matriz da ordem \(3 \times 2\)\\

Desse modo, temos que: 

\[
\begin{aligned}
C_{11} &= 3 \cdot 4 + 0 \cdot 0 = 12,\\[4mm]
C_{12} &= 3 \cdot (-1) + 0 \cdot 2 = -3,\\[4mm]
C_{21} &= (-1) \cdot 4 + 2 \cdot 0 = -4,\\[4mm]
C_{22} &= (-1) \cdot (-1) + 2 \cdot 2 = 1 + 4 = 5,\\[4mm]
C_{31} &= 1 \cdot 4 + 1 \cdot 0 = 4,\\[4mm]
C_{32} &= 1 \cdot (-1) + 1 \cdot 2 = -1 + 2 = 1.
\end{aligned}
\]

Portanto, a matriz \(C\) é
\[
C = \begin{bmatrix}
12 & -3 \\
-4 & 5 \\
4 & 1
\end{bmatrix}.
\]

%---------------------------------------------------------------
%---------------------------------------------------------------

\textbf{Questão 5 item c, pág. 36:}\\

Sejam as matrizes
\[
E = \begin{bmatrix}
6 & 1 & 3 \\
-1 & 1 & 2 \\
4 & 1 & 2
\end{bmatrix}
\quad \text{e} \quad
D = \begin{bmatrix}
1 & 5 & 2 \\
-1 & 0 & 1 \\
3 & 2 & 4
\end{bmatrix}.
\]
\\
Calcula-se o produto \((3E)D\). Primeiro, considere a multiplicação escalar:
\[
3E = \begin{bmatrix}
18 & 3 & 9 \\
-3 & 3 & 6 \\
12 & 3 & 6
\end{bmatrix}.
\]
Em seguida, temos que o produto \((3E)D\) é uma matriz da ordem \(3 \times 3\)\\

\[
\begin{aligned}
X_{11} &= 18\cdot 1 + 3\cdot (-1) + 9\cdot 3 = 18 - 3 + 27 = 42,\\[4mm]
X_{12} &= 18\cdot 5 + 3\cdot 0 + 9\cdot 2 = 90 + 0 + 18 = 108,\\[4mm]
X_{13} &= 18\cdot 2 + 3\cdot 1 + 9\cdot 4 = 36 + 3 + 36 = 75,\\[8mm]
\]
\[
X_{21} &= (-3)\cdot 1 + 3\cdot (-1) + 6\cdot 3 = -3 - 3 + 18 = 12,\\[4mm]
X_{22} &= (-3)\cdot 5 + 3\cdot 0 + 6\cdot 2 = -15 + 0 + 12 = -3,\\[4mm]
X_{23} &= (-3)\cdot 2 + 3\cdot 1 + 6\cdot 4 = -6 + 3 + 24 = 21,\\[8mm]
\]
\[
X_{31} &= 12\cdot 1 + 3\cdot (-1) + 6\cdot 3 = 12 - 3 + 18 = 27,\\[4mm]
X_{32} &= 12\cdot 5 + 3\cdot 0 + 6\cdot 2 = 60 + 0 + 12 = 72,\\[4mm]
X_{33} &= 12\cdot 2 + 3\cdot 1 + 6\cdot 4 = 24 + 3 + 24 = 51.
\end{aligned}
\]

Assim, a matriz resultado é

\[
(3E)D = \begin{bmatrix}
42 & 108 & 75 \\
12 & -3  & 21 \\
27 & 72  & 51
\end{bmatrix}.
\]

%---------------------------------------------------------------
%---------------------------------------------------------------

\textbf{Questão 5 item d, pág. 36:}\\

Conforme calculado no item 5.a, temos
\[
AB = \begin{bmatrix}
12 & -3 \\
-4 & 5 \\
4 & 1
\end{bmatrix}.
\]

Seja
\[
C = \begin{bmatrix}
1 & 4 & 2 \\
3 & 1 & 5
\end{bmatrix}.
\]
Como \(AB\) é uma matriz de ordem \(3 \times 2\) e \(C\) de ordem \(2 \times 3\), o produto resulta em uma matriz \(3 \times 3\).

\[
\begin{aligned}
X_{11} &= 12\cdot 1 + (-3)\cdot 3 = 12 - 9 = 3,\\[4mm]
X_{12} &= 12\cdot 4 + (-3)\cdot 1 = 48 - 3 = 45,\\[4mm]
X_{13} &= 12\cdot 2 + (-3)\cdot 5 = 24 - 15 = 9,\\[8mm]
X_{21} &= (-4)\cdot 1 + 5\cdot 3 = -4 + 15 = 11,\\[4mm]
X_{22} &= (-4)\cdot 4 + 5\cdot 1 = -16 + 5 = -11,\\[4mm]
X_{23} &= (-4)\cdot 2 + 5\cdot 5 = -8 + 25 = 17,\\[8mm]
X_{31} &= 4\cdot 1 + 1\cdot 3 = 4 + 3 = 7,\\[4mm]
X_{32} &= 4\cdot 4 + 1\cdot 1 = 16 + 1 = 17,\\[4mm]
X_{33} &= 4\cdot 2 + 1\cdot 5 = 8 + 5 = 13.
\end{aligned}
\]

Portanto, o produto \((AB)D\) é
\[
(AB)D = \begin{bmatrix}
3 & 45 & 9 \\
11 & -11 & 17 \\
7 & 17 & 13
\end{bmatrix}.
\]

%---------------------------------------------------------------
%---------------------------------------------------------------

\textbf{Questão 7, pág. 36:}\\

Sejam as matrizes
\[
A = \begin{bmatrix}
3 & -2 & 7 \\
6 & 5 & 4 \\
0 & 4 & 9
\end{bmatrix}
\quad \text{e} \quad
B = \begin{bmatrix}
6 & -2 & 4 \\
0 & 1 & 3 \\
7 & 7 & 5
\end{bmatrix}.
\]
\\
Considere as questões a seguir:\\

\textbf{Questão 7 item a, pág. 36:}\\

Dada as matrizes anteriores, podemos definir a primeira linha de \(AB\) como: 

\[
\begin{aligned}
X_{11} &= 18 + 49 = 67,\\[4mm]
X_{12} &= -6 -2 + 49 = 41,\\[4mm]
X_{13} &= 12 -6 + 35 = 41,\\[4mm]
\end{aligned}
\]

Portanto, pode ser definido como:
\[
\begin{bmatrix}
67 & 41 & 41
\end{bmatrix}.
\]

%---------------------------------------------------------------
%---------------------------------------------------------------

\textbf{Questão 7 item b, pág. 36:}\\

Dada as matrizes anteriores, podemos definir a terceira linha de \(AB\) como: 

\[
\begin{aligned}
X_{11} &= 0 + 0 + 63 = 63,\\[4mm]
X_{12} &= 4 + 63 = 67,\\[4mm]
X_{13} &= 12 - 6 + 45 = 57,\\[4mm]
\end{aligned}
\]

Portanto, pode ser definido como:
\[
\begin{bmatrix}
63 & 67 & 57
\end{bmatrix}.
\]

%---------------------------------------------------------------
%---------------------------------------------------------------

\textbf{Questão 7 item c, pág. 36:}\\

Tome a segunda coluna de AB tal que:

\[
\begin{bmatrix}
X_{12}\\
X_{22}\\
X_{32}
\end{bmatrix}.
\]

Note que, dos itens \textbf{a} e \textbf{b}, temos que:

\[
X_{12} &= 41,\\[4mm]
X_{31} &= 67\\[4mm]
\]

Dessa forma, podemos calcular:

\[
X={22} = -12 + 5 + 28 = 21
\]

Assim temos:
\[
\begin{bmatrix}
41\\
21\\
67
\end{bmatrix}.
\]

%---------------------------------------------------------------
%---------------------------------------------------------------

\textbf{Questão 10 item a, pág. 36:}\\

Dadas as linhas e colunas calculadas nos itens \textbf{a, b e c} da \textbf{questão 7}, podemos definir a matriz \(AB\) como: 

\[
AB = \begin{bmatrix}
67 & 41 & 41 \\
X_{21} & 21  & X_{23} \\
63 & 67  & 57
\end{bmatrix}.
\]

Desse modo, resta apenas calcular os itens:

\[
\begin{aligned}
X_{21} &= 36 + 0 + 28 = 64,\\[4mm]
X_{23} &= 24 + 15 + 20 = 59\\[4mm]
\end{bmatrix}.
\]

Desse modo, temos que:
\[
AB = \begin{bmatrix}
67 & 41 & 41 \\
64 & 21  & 59 \\
63 & 67  & 57
\end{bmatrix}.
\]

Agora, expressamos cada coluna de \( AB \) como uma combinação linear das colunas de \( A \):

Primeira coluna:

\[
\begin{bmatrix} 67 \\ 64 \\ 63 \end{bmatrix} =
6 \begin{bmatrix} 3 \\ 6 \\ 0 \end{bmatrix} +
7 \begin{bmatrix} 7 \\ 4 \\ 9 \end{bmatrix}
\]

Segunda coluna:

\[
\begin{bmatrix} 41 \\ 21 \\ 67 \end{bmatrix} =
(-2) \begin{bmatrix} 3 \\ 6 \\ 0 \end{bmatrix} +
1 \begin{bmatrix} -2 \\ 5 \\ 4 \end{bmatrix} +
7 \begin{bmatrix} 7 \\ 4 \\ 9 \end{bmatrix}
\]

Terceira coluna:

\[
\begin{bmatrix} 41 \\ 59 \\ 57 \end{bmatrix} =
4 \begin{bmatrix} 3 \\ 6 \\ 0 \end{bmatrix} +
3 \begin{bmatrix} -2 \\ 5 \\ 4 \end{bmatrix} +
5 \begin{bmatrix} 7 \\ 4 \\ 9 \end{bmatrix}
\]

%---------------------------------------------------------------
%---------------------------------------------------------------

\textbf{Questão 10 item b, pág. 36:}\\

O produto \( BA \) é dado por:

\[
BA = \begin{bmatrix}
6 & -6 & 72 \\
6 & 17 & 31 \\
63 & 41 & 72
\end{bmatrix}.
\]

Agora, expressamos cada coluna de \( BA \) como uma combinação linear das colunas de \( B \):

Primeira coluna:

\[
\begin{bmatrix} 6 \\ 6 \\ 63 \end{bmatrix} =
3 \begin{bmatrix} 6 \\ 0 \\ 7 \end{bmatrix} +
6 \begin{bmatrix} -2 \\ 1 \\ 7 \end{bmatrix} +
0 \begin{bmatrix} 4 \\ 3 \\ 5 \end{bmatrix}
\]

Segunda coluna:

\[
\begin{bmatrix} -6 \\ 17 \\ 41 \end{bmatrix} =
(-2) \begin{bmatrix} 6 \\ 0 \\ 7 \end{bmatrix} +
5 \begin{bmatrix} -2 \\ 1 \\ 7 \end{bmatrix} +
4 \begin{bmatrix} 4 \\ 3 \\ 5 \end{bmatrix}
\]

Terceira coluna:

\[
\begin{bmatrix} 72 \\ 31 \\ 73 \end{bmatrix} =
7 \begin{bmatrix} 6 \\ 0 \\ 7 \end{bmatrix} +
4 \begin{bmatrix} -2 \\ 1 \\ 7 \end{bmatrix} +
9 \begin{bmatrix} 4 \\ 3 \\ 5 \end{bmatrix}
\]

%---------------------------------------------------------------
%---------------------------------------------------------------

\textbf{Questão 11 item a, pág. 36:}\\

\[
\begin{aligned}
2x_1 - 3x_2 + 5x_3 &= 7, \\
9x_1 - x_2 + x_3 &= -1, \\
x_1 + 5x_2 + 4x_3 &= 0.
\end{aligned}
\]

Esse sistema pode ser representado na forma matricial como:

\[
\begin{bmatrix} 
2 & -3 & 5 \\ 
9 & -1 & 1 \\ 
1 & 5 & 4 
\end{bmatrix} 
\begin{bmatrix} 
x_1 \\ 
x_2 \\ 
x_3 
\end{bmatrix} =
\begin{bmatrix} 
7 \\ 
-1 \\ 
0 
\end{bmatrix}.
\]

%---------------------------------------------------------------
%---------------------------------------------------------------

\textbf{Questão 11 item b, pág. 36:}\\

Consideremos o sistema de equações lineares:

\[
\begin{aligned}
4x_1 - 3x_3 + x_4 &= 1, \\
5x_1 + x_2 - 8x_4 &= 3, \\
2x_1 - 5x_2 + 9x_3 - x_4 &= 0, \\
3x_2 - x_3 + 7x_4 &= 2
\end{aligned}
\]

Esse sistema pode ser representado na forma matricial como:

\[
\begin{bmatrix} 
4 & 0 & -3 & 1 \\ 
5 & 1 & 0 & -8 \\ 
2 & -5 & 9 & -1 \\ 
0 & 3 & -1 & 7 
\end{bmatrix} 
\begin{bmatrix} 
x_1 \\ 
x_2 \\ 
x_3 \\ 
x_4
\end{bmatrix} =
\begin{bmatrix} 
1 \\ 
3 \\ 
0 \\ 
2
\end{bmatrix}.
\]

%---------------------------------------------------------------
%---------------------------------------------------------------

\textbf{Questão 13 item a, pág. 36:}\\

Consideremos a seguinte operação matricial:

\[
\begin{bmatrix} 
5 & 6 & -7 \\ 
-1 & -2 & 3 \\ 
0 & 4 & -1 
\end{bmatrix} 
\begin{bmatrix} 
x_1 \\ 
x_2 \\ 
x_3
\end{bmatrix} =
\begin{bmatrix} 
2 \\ 
0 \\ 
3
\end{bmatrix}.
\]

Expandindo essa equação matricial, obtemos o seguinte sistema de equações lineares:

\[
\begin{aligned}
5x_1 + 6x_2 - 7x_3 &= 2, \\
-x_1 - 2x_2 + 3x_3 &= 0, \\
4x_2 - x_3 &= 3.
\end{aligned}
\]

%---------------------------------------------------------------
%---------------------------------------------------------------

\textbf{Questão 13 item b, pág. 36:}\\

Consideremos a seguinte operação matricial:

\[
\begin{bmatrix} 
1 & 1 & 1 \\ 
2 & 3 & 0 \\ 
5 & -3 & -6 
\end{bmatrix} 
\begin{bmatrix} 
x_1 \\ 
x_2 \\ 
x_3
\end{bmatrix} =
\begin{bmatrix} 
2 \\ 
2 \\ 
-9
\end{bmatrix}.
\]

Expandindo essa equação matricial, obtemos o seguinte sistema de equações lineares:

\[
\begin{aligned}
x_1 + x_2 + x_3 &= 2, \\
2x_1 + 3x_2 &= 2, \\
5x_1 - 3x_2 - 6x_3 &= -9.
\end{aligned}
\]

\end{document}