  
\documentclass[a4paper,12pt]{article}
\author{}
\date{}
\usepackage[papersize={216mm,330mm},tmargin=20mm,bmargin=20mm,lmargin=20mm,rmargin=20mm]{geometry}
\usepackage[brazil]{babel}
\usepackage[utf8]{inputenc}
\usepackage{amsmath,amssymb} %,mathabx}%\for eqref
\usepackage{lscape}
\usepackage{graphicx}
\usepackage[colorinlistoftodos]{todonotes}
\usepackage{fancyhdr}
\usepackage{tasks}
\usepackage{float}
\usepackage{multicol}
\usepackage{color}
\usepackage{ragged2e} % justifying
\usepackage{colortbl}
\usepackage{cancel}
\usepackage{stackengine}
\usepackage{mathtools}
\usepackage{tikz}   %create images
\usepackage{tkz-fct}    %create graphics

%====================================================
%============= INFORMACAO DE ENTRADA=================
%====================================================

\def\nomedoaluno{José Manoel Freitas da Silva}
\def\matricula{40028922}

%IMD1002 - Análise Combinatória
%IMD0024 - Cálculo 1
%IMD0034 - Vetores e Geometria Analítica
\def\coddisciplina{IMD0034}
\def\nomedisciplina{Vetores e Geometria Analítica}
\def\codturma{VGA20251T2}
\def\codatividade{U1\_VGATRAB30}

%====================================================
%====================== FIM =========================
%====================================================

\pagestyle{fancy}
\fancyhf{}
\lhead{Aluno: \nomedoaluno\\ Matrícula: \matricula}
\chead{\thepage}
\rhead{Instituto Metrópole Digital - UFRN \\
Turma \codturma}
\lfoot{\nomedisciplina}
\cfoot{Atividade \codatividade}
\rfoot{Prof. Samyr Jácome}

\title{
\vspace{-2cm}
\Large \textbf{Universidade Federal do Rio Grande do Norte}\\ 
Instituto Metrópole Digital \\ 
\coddisciplina $\;$ - \nomedisciplina \\ 
TURMA \codturma \\
\vspace{5mm} \Large\textbf{ATIVIDADE \codatividade} \\
\normalsize Natal-RN, \today\\
\vspace{0.7cm} \large \textit{Prof. Samyr Jácome}\\

\justifying
\vspace{0.5cm} \hspace{-0.82cm}
\begin{minipage}{.6\linewidth}
    \large \textbf{Aluno:} \nomedoaluno
\end{minipage}
\begin{minipage}{.4\linewidth}
    \begin{flushright}
        \large \textbf{Matrícula:} \matricula
    \end{flushright}
\end{minipage}
}

% Tirar a identação do paragrafo no texto
\def\tirarident{\setlength{\parindent}{0cm}} % padrão 15pt.

\setlength{\marginparwidth}{2cm}

%---------------------------------------------------------------
%---------------COMANDO PARA SETA DE ESCALONAMENTO--------------
%---------------------------------------------------------------

\newcommand{\seta}[3][-0.05cm]{%
  \stackon[#1]{
    $\xrightarrow{\mathmakebox[\setawidth]{}}$}{
    \scriptsize{$#2 \,\leftarrow\, #3$}
    }
}

\newcommand{\troca}[3][-0.05cm]{%
  \stackon[#1]{
    $\xrightarrow{\mathmakebox[\setawidth]{}}$}{
    \scriptsize{$#2 \,\leftrightarrow\, #3$}
    }
}

\newlength{\setawidth}% row operation width
\AtBeginDocument{\setlength{\setawidth}{2.0cm}}

%---------------------------------------------------------------
%--------------------------FIM----------------------------------
%---------------------------------------------------------------

\begin{document}
\maketitle

\vspace{-2cm}
\section*{Exercícios}

\tirarident

\textbf{Questão 14, página 56}

\newcommand{\seta}[3][-0.05cm]{%
  \stackon[#1]{
    $\xrightarrow{\mathmakebox[\setawidth]{}}$}{
    \scriptsize{$#2 \,\leftarrow\, #3$}
    }
}

\newcommand{\troca}[3][-0.05cm]{%
  \stackon[#1]{
    $\xrightarrow{\mathmakebox[\setawidth]{}}$}{
    \scriptsize{$#2 \,\leftrightarrow\, #3$}
    }
}

Encontrar a inversa da matriz utilizando o algoritmo da inversão:

\[
A = 
\begin{bmatrix}
1 & 2 & 0 \\
2 & 1 & 2 \\
0 & 2 & 1
\end{bmatrix}
\]

Matriz aumentada:
\[
\left[
\begin{array}{ccc|ccc}
1 & 2 & 0 & 1 & 0 & 0 \\
2 & 1 & 2 & 0 & 1 & 0 \\
0 & 2 & 1 & 0 & 0 & 1
\end{array}
\right]
\]

\subsection*{Passo 1: \seta{L_2}{\frac{L_2 - 2L_1}{-3}}}
\[
\left[
\begin{array}{ccc|ccc}
1 & 2 & 0 & 1 & 0 & 0 \\
0 & 1 & -\tfrac{2}{3} & \tfrac{2}{3} & -\tfrac{1}{3} & 0 \\
0 & 2 & 1 & 0 & 0 & 1
\end{array}
\right]
\]

\subsection*{Passo 2: \seta{L_1}{L_1 - 2L_2}}
\[
\left[
\begin{array}{ccc|ccc}
1 & 0 & \tfrac{4}{3} & -\tfrac{1}{3} & \tfrac{2}{3} & 0 \\
0 & 1 & -\tfrac{2}{3} & \tfrac{2}{3} & -\tfrac{1}{3} & 0 \\
0 & 2 & 1 & 0 & 0 & 1
\end{array}
\right]
\]

\subsection*{Passo 3: \seta{L_3}{\frac{L_3 - 2L_2}{\frac{7}{3}}}}
\[
\left[
\begin{array}{ccc|ccc}
1 & 0 & \tfrac{4}{3} & -\tfrac{1}{3} & \tfrac{2}{3} & 0 \\
0 & 1 & -\tfrac{2}{3} & \tfrac{2}{3} & -\tfrac{1}{3} & 0 \\
0 & 0 & 1 & -\tfrac{4}{7} & \tfrac{2}{7} & \tfrac{3}{7}
\end{array}
\right]
\]

\subsection*{Passo 4: \seta{L_1}{L_1 - \tfrac{4}{3}L_3}}
\[
\left[
\begin{array}{ccc|ccc}
1 & 0 & 0 & \tfrac{3}{7} & \tfrac{2}{7} & -\tfrac{1}{7} \\
0 & 1 & -\tfrac{2}{3} & \tfrac{2}{3} & -\tfrac{1}{3} & 0 \\
0 & 0 & 1 & -\tfrac{4}{7} & \tfrac{2}{7} & \tfrac{3}{7}
\end{array}
\right]
\]

\subsection*{Passo 5: \seta{L_2}{L_2 + \tfrac{2}{3}L_3}}
\[
\left[
\begin{array}{ccc|ccc}
1 & 0 & 0 & \tfrac{3}{7} & \tfrac{2}{7} & -\tfrac{1}{7} \\
0 & 1 & 0 & \tfrac{2}{7} & -\tfrac{1}{7} & \tfrac{2}{7} \\
0 & 0 & 1 & -\tfrac{4}{7} & \tfrac{2}{7} & \tfrac{3}{7}
\end{array}
\right]
\]

Assim, a matriz inversa de $A$ é:
\[
A^{-1} = 
\begin{bmatrix}
\frac{3}{7} & \frac{2}{7} & -\frac{1}{7} \\
\frac{2}{7} & -\frac{1}{7} & \frac{2}{7} \\
-\frac{4}{7} & \frac{2}{7} & \frac{3}{7}
\end{bmatrix}
\]

\[\] 
\textbf{Questão 15, página 56}\\

\[
A = 
\begin{bmatrix}
-1 & 3 & -4 \\
2 & 4 & 1 \\
-4 & 2 & -9
\end{bmatrix}
\]\\

Pelo teorema 1.4.5 do livro, temos que a matriz possui a inversa se e somente se o determinante for diferente de 0. Note que, em uma matriz 3x3, pelo teorema de Sarrus, podemos encontrar o determinante da seguinte forma:
\\

\[
\left[
\begin{array}{cccccc}
-1 & 3 & -4 & -1 & 3 \\
2 & 4 & 1 & 2 & 4 \\
-4 & 2 & -9 & -4 & 2 \\
\end{array}
\right]
\]

Diagonal principal:
\[
(-1)(4)(-9) + (3)(1)(-4) + (-4)(2)(-4) = 36 - 12 + 32 = 56
\]

Diagonal secundária:
\[
(-4)(4)(-4) + (-9)(1)(-1) + (2)(2)(3) = -64 + 1 + 12 = -51
\]

Determinante:
\[
56 - 56 = 0
\]

Como $\det(A) = 0$, a matriz não possui inversa.
\[\] 

\textbf{Questão 03, página 65}\\
Sistema inicial:
\[
\begin{cases}
x_1 + 3x_2 + x_3 = 4 \\
2x_1 + 2x_2 + x_3 = -1 \\
2x_1 + 3x_2 + x_3 = 3
\end{cases}
\]

Forma matricial $AX = B$, onde:
\[
A = 
\begin{bmatrix}
1 & 3 & 1 \\
2 & 2 & 1 \\
2 & 3 & 1
\end{bmatrix}, \quad
X = 
\begin{bmatrix}
x_1 \\ x_2 \\ x_3
\end{bmatrix}, \quad
B = 
\begin{bmatrix}
4 \\ -1 \\ 3
\end{bmatrix}
\]

Matriz aumentada:
\[
\left[
\begin{array}{ccc|ccc}
1 & 3 & 1 & 1 & 0 & 0 \\
2 & 2 & 1 & 0 & 1 & 0 \\
2 & 3 & 1 & 0 & 0 & 1
\end{array}
\right]
\]

\subsection*{Passo 1: \seta{L_2}{\frac{L_2 - 2L_1}{-4}}}
\[
\left[
\begin{array}{ccc|ccc}
1 & 3 & 1 & 1 & 0 & 0 \\
0 & 1 & \tfrac{1}{4} & \tfrac{1}{2} & -\tfrac{1}{4} & 0 \\
2 & 3 & 1 & 0 & 0 & 1
\end{array}
\right]
\]

\subsection*{Passo 2: \seta{L_3}{L_3 - 2L_1}}
\[
\left[
\begin{array}{ccc|ccc}
1 & 3 & 1 & 1 & 0 & 0 \\
0 & 1 & \tfrac{1}{4} & \tfrac{1}{2} & -\tfrac{1}{4} & 0 \\
0 & -3 & -1 & -2 & 0 & 1
\end{array}
\right]
\]

\subsection*{Passo 3: \seta{L_1}{L_1 - 3L_2}}
\[
\left[
\begin{array}{ccc|ccc}
1 & 0 & \tfrac{1}{4} & -\tfrac{1}{2} & \tfrac{3}{4} & 0 \\
0 & 1 & \tfrac{1}{4} & \tfrac{1}{2} & -\tfrac{1}{4} & 0 \\
0 & -3 & -1 & -2 & 0 & 1
\end{array}
\right]
\]

\subsection*{Passo 4: \seta{L_3}{\frac{L_3 + 3L_2}{-1/4}}}
\[
\left[
\begin{array}{ccc|ccc}
1 & 0 & \tfrac{1}{4} & -\tfrac{1}{2} & \tfrac{3}{4} & 0 \\
0 & 1 & \tfrac{1}{4} & \tfrac{1}{2} & -\tfrac{1}{4} & 0 \\
0 & 0 & 1 & 2 & 3 & -4
\end{array}
\right]
\]

\subsection*{Passo 5: \seta{L_1}{L_1 - \tfrac{1}{4}L_3}, \seta{L_2}{L_2 - \tfrac{1}{4}L_3}}
\[
\left[
\begin{array}{ccc|ccc}
1 & 0 & 0 & -1 & 0 & 1 \\
0 & 1 & 0 & 0 & -1 & 1 \\
0 & 0 & 1 & 2 & 3 & -4
\end{array}
\right]
\]

Portanto, a inversa é:
\[
A^{-1} =
\begin{bmatrix}
-1 & 0 & 1 \\
0 & -1 & 1 \\
2 & 3 & -4
\end{bmatrix}
\]

Multiplicando $A^{-1}B$:
\[
X = A^{-1}B =
\begin{bmatrix}
-1 & 0 & 1 \\
0 & -1 & 1 \\
2 & 3 & -4
\end{bmatrix}
\begin{bmatrix}
4 \\ -1 \\ 3
\end{bmatrix}
=
\begin{bmatrix}
-1 \\ 4 \\ -7
\end{bmatrix}
\]

\[
\Rightarrow
x_1 = -1,\quad x_2 = 4,\quad x_3 = -7
\]


\end{document}