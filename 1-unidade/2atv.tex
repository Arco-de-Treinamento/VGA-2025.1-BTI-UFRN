  
\documentclass[a4paper,12pt]{article}
\author{}
\date{}
\usepackage[papersize={216mm,330mm},tmargin=20mm,bmargin=20mm,lmargin=20mm,rmargin=20mm]{geometry}
\usepackage[brazil]{babel}
\usepackage[utf8]{inputenc}
\usepackage{amsmath,amssymb} %,mathabx}%\for eqref
\usepackage{lscape}
\usepackage{graphicx}
\usepackage[colorinlistoftodos]{todonotes}
\usepackage{fancyhdr}
\usepackage{tasks}
\usepackage{float}
\usepackage{multicol}
\usepackage{color}
\usepackage{ragged2e} % justifying
\usepackage{colortbl}
\usepackage{cancel}
\usepackage{stackengine}
\usepackage{mathtools}
\usepackage{tikz}   %create images
\usepackage{tkz-fct}    %create graphics

%====================================================
%============= INFORMACAO DE ENTRADA=================
%====================================================

\def\nomedoaluno{José Manoel Freitas da Silva}
\def\matricula{40028922}

%IMD1002 - Análise Combinatória
%IMD0024 - Cálculo 1
%IMD0034 - Vetores e Geometria Analítica
\def\coddisciplina{IMD0034}
\def\nomedisciplina{Vetores e Geometria Analítica}
\def\codturma{VGA20251T2}
\def\codatividade{U1\_VGATRAB20}

%====================================================
%====================== FIM =========================
%====================================================

\pagestyle{fancy}
\fancyhf{}
\lhead{Aluno: \nomedoaluno\\ Matrícula: \matricula}
\chead{\thepage}
\rhead{Instituto Metrópole Digital - UFRN \\
Turma \codturma}
\lfoot{\nomedisciplina}
\cfoot{Atividade \codatividade}
\rfoot{Prof. Samyr Jácome}

\title{
\vspace{-2cm}
\Large \textbf{Universidade Federal do Rio Grande do Norte}\\ 
Instituto Metrópole Digital \\ 
\coddisciplina $\;$ - \nomedisciplina \\ 
TURMA \codturma \\
\vspace{5mm} \Large\textbf{ATIVIDADE \codatividade} \\
\normalsize Natal-RN, \today\\
\vspace{0.7cm} \large \textit{Prof. Samyr Jácome}\\

\justifying
\vspace{0.5cm} \hspace{-0.82cm}
\begin{minipage}{.6\linewidth}
    \large \textbf{Aluno:} \nomedoaluno
\end{minipage}
\begin{minipage}{.4\linewidth}
    \begin{flushright}
        \large \textbf{Matrícula:} \matricula
    \end{flushright}
\end{minipage}
}

% Tirar a identação do paragrafo no texto
\def\tirarident{\setlength{\parindent}{0cm}} % padrão 15pt.

\setlength{\marginparwidth}{2cm}

%---------------------------------------------------------------
%---------------COMANDO PARA SETA DE ESCALONAMENTO--------------
%---------------------------------------------------------------

\newcommand{\seta}[3][-0.05cm]{%
  \stackon[#1]{
    $\xrightarrow{\mathmakebox[\setawidth]{}}$}{
    \scriptsize{$#2 \,\leftarrow\, #3$}
    }
}

\newcommand{\troca}[3][-0.05cm]{%
  \stackon[#1]{
    $\xrightarrow{\mathmakebox[\setawidth]{}}$}{
    \scriptsize{$#2 \,\leftrightarrow\, #3$}
    }
}

\newlength{\setawidth}% row operation width
\AtBeginDocument{\setlength{\setawidth}{2.0cm}}

%---------------------------------------------------------------
%--------------------------FIM----------------------------------
%---------------------------------------------------------------

\begin{document}
\maketitle

\vspace{-2cm}
\section*{Exercícios}

\tirarident

\textbf{Questão 7, pág. 23:}\\

Dado o sistema de equações lineares:
\[
\begin{cases}
x - y + 2z - w = -1 \\
2x + y - 2z - 2w = -2 \\
- x + 2y - 4z + w = 1 \\
3x - 3w = -3
\end{cases}
\]

Podemos representá-lo na forma matricial aumentada:
\[
A =
\begin{bmatrix}
1 & -1 & 2 & -1 & -1 \\
2 & 1 & -2 & -2 & -2 \\
-1 & 2 & -4 & 1 & 1 \\
3 & 0 & 0 & -3 & -3
\end{bmatrix}
\]

Aplicamos as seguintes operações:

1. \( A_2 \leftarrow A_2 - 2A_1 \)
\[
\begin{bmatrix}
1 & -1 & 2 & -1 & -1 \\
0 & 3 & -6 & 0 & 0 \\
-1 & 2 & -4 & 1 & 1 \\
3 & 0 & 0 & -3 & -3
\end{bmatrix}
\]

2. \( A_3 \leftarrow A_3 + A_1 \)
\[
\begin{bmatrix}
1 & -1 & 2 & -1 & -1 \\
0 & 3 & -6 & 0 & 0 \\
0 & 1 & -2 & 0 & 0 \\
3 & 0 & 0 & -3 & -3
\end{bmatrix}
\]

3. \( A_4 \leftarrow A_4 - 3A_1 \)
\[
\begin{bmatrix}
1 & -1 & 2 & -1 & -1 \\
0 & 3 & -6 & 0 & 0 \\
0 & 1 & -2 & 0 & 0 \\
0 & 3 & -6 & 0 & 0
\end{bmatrix}
\]

4. Permutamos \( A_2 \) com \( A_3 \):
\[
\begin{bmatrix}
1 & -1 & 2 & -1 & -1 \\
0 & 1 & -2 & 0 & 0 \\
0 & 3 & -6 & 0 & 0 \\
0 & 3 & 6 & 0 & 0
\end{bmatrix}
\]

5. \( A_1 \leftarrow A_1 + A_2 \)
\[
\begin{bmatrix}
1 & 0 & 0 & -1 & -1 \\
0 & 1 & -2 & 0 & 0 \\
0 & 3 & -6 & 0 & 0 \\
0 & 3 & -6 & 0 & 0
\end{bmatrix}
\]

6. \( A_3 \leftarrow A_3 - 3A_2 \) e \( A_4 \leftarrow A_4 - 3A_2 \)
\[
\begin{bmatrix}
1 & 0 & 0 & -1 & -1 \\
0 & 1 & -2 & 0 & 0 \\
0 & 0 & 0 & 0 & 0 \\
0 & 0 & 0 & 0 & 0
\end{bmatrix}
\]

Assim, obtemos as equações reduzidas:
\[
\begin{cases}
x - w = -1 \Rightarrow x = -1 + w \\
y - 2z = 0 \Rightarrow y = 2z
\end{cases}
\]


%---------------------------------------------------------------
%---------------------------------------------------------------

\textbf{Questão 9, pág. 23:}\\

Dado o sistema de equações:
\[
\begin{cases}
x_1 + x_2 + 2x_3 = 8 \\
- x_1 - 2x_2 + 3x_3 = 1 \\
-3x_1 - 7x_2 + 4x_3 = 10
\end{cases}
\]

Representamos na forma matricial aumentada:
\[
A =
\begin{bmatrix}
1 & 1 & 2 & 8 \\
-1 & -2 & 3 & 1 \\
-3 & -7 & 4 & 10
\end{bmatrix}
\]

Aplicamos as seguintes transformações:

1. \( f_1: A_2 \leftarrow (A_2 - A_1) \cdot (-1) \)
\[
\begin{bmatrix}
1 & 1 & 2 & 8 \\
0 & 1 & -5 & -9 \\
-3 & -7 & 4 & 10
\end{bmatrix}
\]

2. \( f_2: A_3 \leftarrow A_3 + 3A_1 \)
\[
\begin{bmatrix}
1 & 1 & 2 & 8 \\
0 & 1 & -5 & -9 \\
0 & -4 & 10 & 34
\end{bmatrix}
\]

3. \( f_3: A_3 \leftarrow \frac{A_3 -(- 4A_2)}{-10} \)
\[
\begin{bmatrix}
1 & 1 & 2 & 8 \\
0 & 1 & -5 & -9 \\
0 & 0 & 1 & \frac{1}{5}
\end{bmatrix}
\]

O sistema escalonado resultante é:
\[
\begin{cases}
x_1 + x_2 + 2x_3 = 8 \\
x_2 - 5x_3 = -9 \\
x_3 = \frac{1}{5}
\end{cases}
\]


%---------------------------------------------------------------
%---------------------------------------------------------------

\textbf{Questão 19, pág. 23:}\\

Dado o sistema de equações:
\[
\begin{cases}
3x_1 + x_2 + x_3 + x_4 = 0 \\
5x_1 - x_2 + x_3 - x_4 = 0
\end{cases}
\]

Representamos na forma matricial aumentada:
\[
A =
\begin{bmatrix}
3 & 1 & 1 & 1 & 0 \\
5 & -1 & 1 & -1 & 0
\end{bmatrix}
\]

Aplicamos as seguintes transformações:

1. \( f_1: A_1 \leftarrow \frac{A_1}{3} \)
\[
\begin{bmatrix}
1 & \frac{1}{3} & \frac{1}{3} & \frac{1}{3} & 0 \\
5 & -1 & 1 & -1 & 0
\end{bmatrix}
\]

2. \( f_2: A_2 \leftarrow A_2 - 5A_1 \)
\[
\begin{bmatrix}
1 & \frac{1}{3} & \frac{1}{3} & \frac{1}{3} & 0 \\
0 & -\frac{8}{3} & -\frac{2}{3} & -\frac{8}{3} & 0
\end{bmatrix}
\]

3. \( f_3: A_2 \leftarrow \frac{A_2}{-8/3} \)
\[
\begin{bmatrix}
1 & \frac{1}{3} & \frac{1}{3} & \frac{1}{3} & 0 \\
0 & 1 & \frac{1}{4} & 1 & 0
\end{bmatrix}
\]

4. \( f_4: A_1 \leftarrow A_1 - \frac{1}{3}A_2 \)
\[
\begin{bmatrix}
1 & 0 & \frac{1}{4} & 0 & 0 \\
0 & 1 & \frac{1}{4} & 1 & 0
\end{bmatrix}
\]

O sistema escalonado resultante é:
\[
\begin{cases}
x_1 + \frac{x_3}{4} = 0 \\
x_2 + \frac{x_3}{4} + x_4 = 0
\end{cases}
\]

Isolando as variáveis, temos que:
\[
x_1 = -\frac{x_3}{4}, \quad x_2 = x_1 - x_4
\]

%---------------------------------------------------------------
%---------------------------------------------------------------

\textbf{Questão 32, pág. 24:}\\

%Virou sudoku agora ??

Seja a matriz inicial:

\[
A =
\begin{bmatrix}
2 & 1 & 3 \\
0 & -2 & -29 \\
3 & 4 & 5
\end{bmatrix}
\]

1. \( f: A_1 \leftarrow 5A_1 \), \( A_3 \leftarrow 5A_3 \)

\[
\begin{bmatrix}
10 & 5 & 15 \\
0 & -2 & -29 \\
15 & 20 & 25
\end{bmatrix}
\]

2. \( f: A_3 \leftarrow \frac{A_3 - A_1}{5} \)

\[
\begin{bmatrix}
10 & 5 & 15 \\
0 & -2 & -29 \\
1 & 3 & 2
\end{bmatrix}
\]

3. Permutamos \( A_3 \) com \( A_1 \):

\[
\begin{bmatrix}
1 & 3 & 2 \\
0 & -2 & -29 \\
10 & 5 & 15
\end{bmatrix}
\]

4. \( f: A_3 \leftarrow \frac{A_3 * A_2 * A_1}{-30} \)

\[
\begin{bmatrix}
1 & 3 & 2 \\
0 & -2 & -29 \\
0 & 1 & 29
\end{bmatrix}
\]

5. Permutamos \( A_3 \) com \( A_2 \):

\[
\begin{bmatrix}
1 & 3 & 2 \\
0 & 1 & 29 \\
0 & -2 & -29
\end{bmatrix}
\]

O sistema resultante é:

\[
\begin{cases}
x + 3y = 2 \\
y = 29 \\
-2y = -29 \Rightarrow 2y = 29
\end{cases}
\]

Observe que \( y = 29 \) e \( 2y = 29 \), o que é um absurdo. Portanto, o resultado é inconclusivo.

\end{document}